\documentclass{report}

% Pour les chapitres sans numéro
\newcommand{\mychapter}[2]{
    \setcounter{chapter}{0}
    \setcounter{section}{0}
    \chapter*{#2}
    \addcontentsline{toc}{chapter}{#2}
}

\usepackage[utf8]{inputenc}
\usepackage[T1]{fontenc}
\usepackage[francais]{babel}
\usepackage{hyperref}

\title{%
    \begin{minipage}\linewidth
        \centering
        RAPPORT DE PROJET
        \linebreak
        \large "Cloud et virtualisation avec Openstack"
    \end{minipage}
}
\author{%
    \begin{minipage}\linewidth
        \centering
        Tutrice : BOUZIANE HINDE\linebreak
        \linebreak
        \large Groupe : CULTY Alexandre, BENAIS Charles, BRESSAND Jérémy, ROGLIANO Théo
    \end{minipage}
}
\date{2015 - 2016}

\begin{document}

\maketitle % Page de garde


% ------------------------------ INTRODUCTION ------------------------------
\newpage
\mychapter{1}{Introduction}
\section{Présentation du cloud}
    \large{
        Lorem ipsum dolor sit amet, consectetur adipiscing elit. Quisque ante ipsum,
        congue semper nisi ac, fermentum sagittis lorem. Aliquam egestas, elit vitae
        ultrices mattis, felis ipsum euismod nibh, nec tristique metus ante at dolor.
        Curabitur dapibus metus ut felis gravida viverra. Quisque vel sapien tellus.
        Etiam nec molestie magna. Curabitur ac placerat eros. Praesent eget risus
        condimentum, viverra augue et, rutrum erat. Maecenas in justo tincidunt,
        luctus massa nec, egestas diam. In cursus dignissim elit, at bibendum nunc
        pretium eu. Integer imperdiet aliquet gravida.
}


% ------------------------------ CONTENU ------------------------------
\newpage
\section{Présentation des domaines}
Réseau, Virtualisation, Administration, Informatique distribuée, IaaS,
OpenStack, Puppet, Scripting, 

\section{Présentation du problème}
\subsection{Mise en place}
\subsubsection{Single machine}
Dans le but de nous familiariser avec ce nouvel environnement,
et l'accès à des machines possédant les droits administrateurs (sudo)
ne nous étant pas offert à la faculté des sciences,
nous avons choisi de mettre en place OpenStack sur nos machines personnelles.\break
Devstack est un script bash automatisant la mise en place d'un environnement OpenStack 
sur une grande variété de machines; mis en place par la communauté, 
DevStack à pour objectif de fournir un 'quick operational testing', une démonstration 
d'OpenStack et de ses fonctionnalités.

\subsubsection{Multi-nodes}
Dans un second temps il nous à été offert l'opportunité de travailler sur les clusters
de Grid5000, un Réseau de Ressources distribuées supporté par l'INRIA et le CNRS.\break

\section{Description du travail}
Automatisation de l'installation de logiciel sur un parc de VM OpenStack via Puppet

% ------------------------------ CONCLUSION ------------------------------
\newpage
\mychapter{0}{Conclusion}


% ------------------------------ REMERCIEMENTS ------------------------------
\newpage
\mychapter{0}{Remerciements}
\large{
    Nous tenons tout d'abord à remercier madame BOUZIANE Hinde, notre tutrice de projet, qui nous a permis de réaliser ce projet.\linebreak

    Nous voulons aussi remercier l'équipe de Grid5000 qui nous à permis de réaliser ce projet sur leur système en nous donnant accès à leurs serveurs.
}

% ------------------------------ RESSOURCES DOCUMENTAIREs ------------------------------
\newpage
\mychapter{0}{Ressources documentaires}
\href{http://docs.openstack.org/developer/devstack/}{Devstack} :
\url{http://docs.openstack.org/developer/devstack/}
\bigbreak
\href{http://www.openstack.org/}{Openstack} :
\url{http://www.openstack.org/}
\bigbreak
\href{https://puppetlabs.com/puppet/what-is-puppet}{Puppet} :
\url{https://puppetlabs.com/puppet/what-is-puppet}
\bigbreak
\href{https://www.grid5000.fr/}{Grid5000} :
\url{https://www.grid5000.fr/}
\bigbreak
\href{https://github.com/grid5000/xp5k-openstack}{XP5K} :
\url{https://github.com/grid5000/xp5k-openstack}
\end{document}
